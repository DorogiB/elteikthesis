\chapter{Bevezetés}
\label{ch:intro}

Az informatikai projekteknek a munkafolyamatok követése, a különböző állapotok megtekintésének és visszaállításának lehetősége kulcsfontosságú részét képezi az agilis projektmenedzsment, valamint a kooperatív és elosztott munkavégzés elősegítése érdekében. Napjainkban a szöveges alapú munkák -- például szoftverkód, szöveges dokumentumok, táblázatok -- esetében már számtalan eszköz áll rendelkezésünkre ezen funkcionalitás eléréséhez, a már elavultnak számító -- de ennek ellenére még mindig használt -- CVS és SVN szoftverektől, a jelenleg alapvetőnek számító git alapú rendszerekig.\todo[inline]{Azért azt erősnek érzem, hogy az SVN elavultnak számítana, már csak azért is, mert egyes felmérések szerint még mindig ez a legnépszerűbb verziókezelő rendszer. A "git alapú" megfogalmazás sem szerencsés, mert az alapokat nem a Git fektette le, csak az lett népszerű; nagy részben a GitHub miatt. Javaslom inkább a centrális és elosztott elvű verziókezelési fogalmakat használni, a konkrét termékek (Git, SVN) pedig inkább csak példák legyenek.}
Az említett verziókezelők közös tulajdonsága, hogy mind úgynevezett állapot alapú revíziókezelést valósítanak meg, azaz minden egyes módosításkor az állományok nyers változásait, sorok törlését, hozzáadását tárolják, és ezek alapján állítanak elő újabb vagy régebbi revíziókat.

Az állapot alapú verziókezelés egyik hiányossága, hogy az adatokat binárisan tároló folyamatok esetében -- amik jellemzően valamilyen grafikus információt tárolnak vektorosan --, nagy tárigényű és jóval kevésbé hatékony.
\todo[inline]{Vannak azért raszteres bináris állományok is. Fontos lenne hangsúlyozni, hogy vektoros állományok esetén az állapot alapú verziókezelés elveszti a módosítás szemantikai információját, hogy pontosan milyen transzformációs művelet került végrehajtásra. Ez a szöveges vektoros állományokra is igaz, pl. SVG. Bináris állományok esetén pedig a tárolás sem túl hatékony természetesen.}
Az ilyen téradatok esetében a művelet alapú revíziókezelés sokkal célravezetőbb, mivel ebben az esetben a változások (vagy más szóval \emph{delták}) az adott objektumokon végzett műveletek, melyek újbóli végrehajtásával megkaphatjuk a frissített változatát a módosított adatnak, az invertált műveletek alkalmazásával pedig korábbi állapotokat érhetünk el.
A téradatok hatékony verziókezelésére való igényre válaszként született az Eötvös Loránd Tudományegyetemen fejlesztett \emph{AEGIS} geoinformatikai programcsomag \cite{giachetta2013aegis, aegis} revíziókezelő modulja \cite{cserep2013versioning,cserep2015operation}, ami interfészt biztosít az összes szükséges folyamathoz, viszont ezen szoftvernek még nem született valós, ipari környezetben is felhasználható implementációja.

Szakdolgozatom motivációja az előbbi hiány megszüntetése, a széleskörűen elterjedt és használt nyílt forráskódú \emph{QGIS} térinformatikai programhoz \cite{qgis} írt verziókezelő beépülő modul megvalósításával, mely a \emph{Q-Aegis} nevet kapta.

A modul feladata a QGIS alapvető geometriai műveleteinek naplózása, tetszőleges verzióállapotok betöltése az AEGIS könyvtár funkcióinak felhasználásával

%A dolgozat 2. fejezete telepítési és használati útmutatóként szolgál a szoftver használatához, a 3. fejezet áttekinti a program technikai implementációjának részleteit, kitérve az esetleg furcsának tűnő megoldások magyarázataira, az utolsó fejezet pedig összegzi a feladat megoldásának folyamatát, és bemutatja az esetleges továbbfejlesztési lehetőségeket is.


%Lorem ipsum dolor sit amet, consectetur adipiscing elit. In eu egestas mauris. Quisque nisl elit, varius in erat eu, dictum commodo lorem. Sed commodo libero et sem laoreet consectetur. Fusce ligula arcu, vestibulum et sodales vel, venenatis at velit \cite{dahl1972structured}. Aliquam erat volutpat. Proin condimentum accumsan velit id hendrerit. Cras egestas arcu quis felis placerat, ut sodales velit malesuada. Maecenas et turpis eu turpis placerat euismod.\footnote{Maecenas a urna viverra, scelerisque nibh ut, malesuada ex.}

%Aliquam suscipit dignissim tempor. Praesent tortor libero, feugiat et tellus porttitor, malesuada eleifend felis. Orci varius natoque penatibus et magnis dis parturient montes, nascetur ridiculus mus \cite{cormen2009algorithms,krasner1988mvc}. Nullam eleifend imperdiet lorem, sit amet imperdiet metus pellentesque vitae. Donec nec ligula urna. Aliquam bibendum tempor diam, sed lacinia eros dapibus id. Donec sed vehicula turpis. Aliquam hendrerit sed nulla vitae convallis. Etiam libero quam, pharetra ac est nec, sodales placerat augue. \citeauthor{dijkstra1979goto} praesent eu consequat purus \cite{dijkstra1979goto}. 
