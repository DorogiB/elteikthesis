\chapter{Bevezetés}
\label{ch:intro}

Az informatikai projekteknek a munkafolyamatok követése, a különböző állapotok megtekintésének és visszaállításának lehetősége kulcsfontosságú részét képezi az agilis projektmenedzsment, valamint a kooperatív és elosztott munkavégzés elősegítése érdekében. Napjainkban a szöveges alapú munkák -- például szoftverkód, szöveges dokumentumok, táblázatok -- esetében már számtalan eszköz áll rendelkezésünkre ezen funkcionalitás eléréséhez. Attól függően, hogy milyen módon tárolják a kezelt állományokat, a verziókezelő rendszereket két részre oszthatjuk. A centralizált verziókezelők egy központi tárhelyen tárolják a fájlokat, és minden felhasználó ezt a központi verziót módosíthatja, ilyen elven működik például az SVN. Az elosztott revíziókezelők esetében minden felhasználó rendelkezik a követett projekt teljes másolatával, és szerkesztéskor ez módosul, ilyen rendszerre a legismertebb példa a Git.
Az említett verziókezelők közös tulajdonsága, hogy mind úgynevezett állapot alapú revíziókezelést valósítanak meg, azaz minden egyes módosításkor az állományok nyers változásait, sorok törlését, hozzáadását tárolják, és ezek alapján állítanak elő újabb vagy régebbi revíziókat.

Az állapot alapú verziókezelés egyik hiányossága, hogy az adatokat binárisan tároló folyamatok esetében -- amik jellemzően valamilyen grafikus információt tárolnak --, nagy tárigényű és jóval kevésbé hatékony. Ezen kívül problémát jelent az is, hogy az állapot alapú verziókezelés esetén a módosítások szemantikai információja elvész, nem követhető, hogy pontosan milyen transzformációk kerültek végrehajtásra, függetlenül a tárolás módjától.
Az ilyen téradatok esetében a művelet alapú revíziókezelés sokkal célravezetőbb, mivel ebben az esetben a változások (vagy más szóval \emph{delták}) az adott objektumokon végzett műveletek, melyek újbóli végrehajtásával megkaphatjuk a frissített változatát a módosított adatnak, az invertált műveletek alkalmazásával pedig korábbi állapotokat érhetünk el.
A téradatok hatékony verziókezelésére való igényre válaszként született az Eötvös Loránd Tudományegyetemen fejlesztett \emph{AEGIS} geoinformatikai programcsomag \cite{giachetta2013aegis, aegis} revíziókezelő modulja \cite{cserep2013versioning,cserep2015operation}, ami interfészt biztosít az összes szükséges folyamathoz, viszont ezen szoftvernek még nem született valós, ipari környezetben is felhasználható implementációja.

Szakdolgozatom motivációja az előbbi hiány megszüntetése, a széleskörűen elterjedt és használt nyílt forráskódú \emph{QGIS} térinformatikai programhoz \cite{qgis} írt verziókezelő beépülő modul megvalósításával, mely a \emph{Q-Aegis} nevet kapta.

A modul feladata a \emph{QGIS} alapvető geometriai műveleteinek naplózása, tetszőleges verzióállapotok betöltése az \emph{AEGIS} könyvtár funkcióinak felhasználásával

